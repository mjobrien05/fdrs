\documentclass[11pt]{article}
\usepackage{amsthm}
\usepackage{amsmath}
\usepackage{amssymb}
\usepackage{graphicx}    % needed for including graphics e.g. EPS, PS
\textheight 23.4cm \textwidth 14.65cm 
\oddsidemargin 0.375in 
\evensidemargin 0.375in 
\topmargin  -0.55in
\pagestyle{empty}       % Uncomment if don't want page numbers

\newcommand{\inv}{^{-1}}
\newcommand{\half}{^{\frac 12}}
\newcommand{\plus}{^+}
\newcommand{\minus}{^-}
\newcommand{\inner}[2]{\langle #1, #2\rangle}
\newcommand{\argmin}[1]{\underset{#1}{\operatorname{arg\,min}}\;}
\newcommand{\sub}{\partial}
\def\grad {{\nabla}}

\usepackage{cleveref} % must come before algorithm, for some reason
\usepackage{algorithm}
\usepackage{algorithmic}

\newtheorem{thrm}{Theorem}
\newtheorem{lemma}{Lemma}
\newtheorem{prb}{Problem}
\newtheorem{assum}{Assumption}

\crefname{algocf}{Algorithm}{Algorithms}
\Crefname{algocf}{Algorithm}{Algorithms}
\crefname{Figure}{Figure}{Figures}
\Crefname{figure}{Figure}{Figures}
\crefname{equation}{equation}{equations}
\Crefname{equation}{Equation}{Equations}
\crefname{assum}{assumption}{assumptions}
\Crefname{assum}{Assumption}{Assumptions}
\crefname{prb}{problem}{problems}
\Crefname{prb}{Problem}{Problems}
\crefname{lemma}{Lemma}{Lemma}
\Crefname{lemma}{Lemma}{Lemmas}

\begin{document}
\title{An Acceleration of the Douglas Rachford Splitting Algorithm}
\maketitle
\tableofcontents
\begin{abstract}
The bullets are from Tom's outline, now incorporated within the main document.
\end{abstract}
\section{Introduction}
\label{sec:intro}


\begin{itemize}
\item Why do we need splitting methods
\item Common problem in imaging require large numberd of unknown
\item ``Big Data'' applications have large number of unknowns
\item We need methods that can handle sophisitcated problems, yet have ``cheap'' steps
\end{itemize}

%Define the subgradient of $f$ as 

%\begin{align}
%\sub f(z) = \{p \in \mathcal{R}^n : f(v) \ge f(z) + \inner{p}{v-z} \;\forall v \in \mathcal{R}^n\}
%\end{align}

%A convex function $f$ lies above its tangent line:

%\begin{align}
%f(y) \ge f(x) + \inner {\sub f(x)}{y-x}, \label{prop:convex}
%\end{align}

%\noindent
%where $\sub$ is the subdifferential operator, or the set of all
%subgradients of $f$.  If $f$ is differentiable with gradient $\grad
%f$, then $\grad f = \sub f$.

\noindent This manuscript considers the problem 

\begin{prb}
minimize $F(\lambda) = H(\lambda) + G(\lambda)$, \label{prb:1}
\end{prb}

\noindent for convex $H$ and $G$, where $\lambda \in \mathbb{R}^N$.
In this manuscript, we will focus on solving \cref{prb:1} with the
additional given:

\begin{assum}
\emph{$H$ and $G$ have a Lipschitz continuous gradients, with Lipschitz constants $L(\grad H) = \sigma_H$ and $L(\grad G) = \sigma_G$.} \label{assum:1}
\end{assum}

\noindent With this additional assumption, we can prove a stronger
convergent result for an accelerated Douglas Rachford splitting
algorithm.  Before we introduce the algorithm, we introduce a few
important properties and definitions.

\subsection{Terminology and Notation}

\begin{itemize}
\item Define notation for norms, etc..
\item Define strongly convex, convex conjugate, Lipschitz constant, and whatever else we use
\end{itemize}

If a function $f$ has a Lipschitz continuous gradient with Lipschitz
constant $\sigma$, then

\begin{align}
f(y) \le f(x) + \inner{\grad f(x)}{y-x} + \frac {\sigma}2 ||y-x||^2. \label{prop:lipschitzcontgrad}
\end{align}

\subsection{Splitting Methods}

\subsubsection{Douglas Rachford Splitting}
\begin{itemize}
\item History and references of inventors
\item Introduce different forms of the method, and show their equivalence (e.g. repeat stuff from Ernie's paper)
\item Mention that the method is also equivalent to ADMM
\item State ADMM algortihm and discuss equivalence
\end{itemize}

\noindent Consider the following algorithm for finding a solution
$\lambda^\star = \argmin \lambda F(\lambda)$, called Douglas-Rachford
Splitting (DRS).  In order to minimize $F(\lambda) = H(\lambda) +
G(\lambda)$, we consider the problem of solving

\begin{align}
\frac {d\lambda}{dt} + \grad H(\lambda) + \grad G(\lambda) = 0.
\end{align}

\noindent Then, iterating

\begin{align}
&\frac {\lambda_{k+1/2} - \lambda_k}\tau + \grad H(\lambda_{k+1/2}) + \grad G(\lambda_k) = 0 \\
&\frac {\lambda_{k+1} - \lambda_k}\tau + \grad H(\lambda_{k+1/2}) + \grad G(\lambda_{k+1}) = 0
\end{align}

\noindent to steady state produces a solution to $\grad H(\lambda) +
\grad G(\lambda)$.  Rearranging these iterates, we arrive at

\begin{align}
&\lambda_{k+1/2} + \tau\grad H(\lambda_{k+1/2}) = \lambda_k - \tau\grad G(\lambda_k) \label{eq:derivation:1}\\
&\lambda_{k+1} + \tau\grad G(\lambda_{k+1}) = \lambda_k - \tau\grad H(\lambda_{k+1/2}). \label{eq:derivation:2}
\end{align}

\noindent The \cref{eq:derivation:1} can be written as

\begin{align}
\lambda_{k+1/2} = J_{\tau\grad H}(\lambda_k - \tau\grad G(\lambda_k)), \label{eq:drsprox:1}
\end{align}

\noindent and \cref{eq:derivation:2} is equivalent to

\begin{align}
\lambda_{k+1} &= J_{\tau\grad G}(\lambda_k - \tau\grad H(\lambda_{k+1/2})). \label{eq:drsprox:2}
\end{align}

\noindent The steps in \cref{eq:drsprox:1,eq:drsprox:2} form the core
subroutine of the Douglas Rachford algorithms, which is given in
\cref{alg:DRSstep}.

\begin{algorithm}[!]
\caption{$\texttt{DRSstep}(\lambda\minus)$}
\label{alg:DRSstep}
\begin{algorithmic}[1]
\REQUIRE $\lambda\minus$
\STATE $\lambda\half = J_{\tau \grad H}(\lambda\minus - \tau \grad G(\lambda\minus)) = \argmin{\lambda} \tau H(\lambda) + \frac 12||\lambda - \lambda\minus + \tau \grad G(\lambda\minus)||^2$
\STATE $\lambda\plus = J_{\tau \grad G}(\lambda\half + \tau \grad G(\lambda\minus)) = \argmin{\lambda} \tau G(\lambda) + \frac 12||\lambda - \lambda\half - \tau \grad G(\lambda\minus)||^2$
\RETURN {$\lambda\plus$}
\end{algorithmic}
\end{algorithm}

\noindent Douglas Rachford splitting is then prescribed by
\cref{alg:drs}:

\begin{algorithm}[!]
\caption{DRS}
\label{alg:drs}
\begin{algorithmic}[1]
\REQUIRE $\lambda_0 \in R^N$
\FOR{$k = 0,1,2,\ldots$}
  \STATE $\lambda_{k+1} = \texttt{DRSstep}(\lambda_k)$
\ENDFOR
\end{algorithmic}
\end{algorithm}

\subsection{Optimality Conditions and Residuals}

The optimality conditions for \cref{alg:DRSstep} are:

\begin{align}
\lambda\minus - \lambda\half &= \tau \left(\grad G(\lambda\minus) + \grad H(\lambda\half) \right) \label{opt:drs:1}\\
\lambda\plus - \lambda\half &= \tau \left(\grad G(\lambda\minus) - \grad G(\lambda\plus)\right) \label{opt:drs:2}
\end{align}

\noindent The residual for \cref{alg:drs} is

\begin{align}
r_k = ||\grad G(\lambda_k) + \grad H(\lambda_k)|| \label{eq:res}.
\end{align}


\section{Optimal Splitting Methods}
Prior Work:  State methods and give references.

\subsection{Nesterov Acceleration}
\begin{itemize}
\item Give some history:  Gradient methods are very effective for simple problems, but they converge slow, and Nesterov invented a way to make them fast
\item State gradient descent method.  This method has $O(1/k)$ complexity.
\item State Nesterov method, this has $O(1/k^2)$ complexity
\end{itemize}

\subsection{Guler's Fast Proximal Point}
\subsection{FISTA}
\subsection{Goldfarb's Fast ADMM}
\subsection{Tom's Fast ADMM}
\subsection{FDRS}
State that there's no known fast DRS.  We propose a new fast DRS.  the approach will follow Tom's approach for ADMM, but by working on DRS, we will prove stronger convergence results.


\section{Preliminary Results}

For the arguments presented here, we will leverage \cref{alg:DRSstep},
and use the corresponding notation.  We will also need the following
alternate expression, which follows from
\cref{eq:derivation:1,eq:drsprox:2}:

\begin{align}
\begin{split}
\lambda\plus &= J_{\tau\grad G}(\lambda\minus - \tau\grad H(\lambda\half)) \\
&= \argmin{\lambda} \tau G(\lambda) + \frac 12||\lambda - \lambda\minus + \tau \grad H(\lambda\half)||^2.  \label{eq:drsprox:2:alt} 
\end{split}
\end{align}

\noindent From \cref{eq:drsprox:2:alt}, we can derive an alternate (and equivalent) optimality condition for the second step of \cref{alg:DRSstep}:

\begin{align}
\lambda\minus - \lambda\plus &= \tau (\grad G(\lambda\plus) + \grad H(\lambda\half)) \label{opt:drs:2:alt}
\end{align}


\noindent Tom says: In this first result [this coming lemma], don't assume that
the energy is differentiable.  This lemma will hold if we either (a)
use a line search to guarantee inequalities, or (b) have Lipschitz
constants for the gradients.

\begin{lemma}
\label{lem:1}
Suppose that $\tau^3 \le \frac 1{\sigma_H\sigma_G^2}$.  Then for any $\gamma \in R^N$,
\begin{align*}
F(\gamma) - F(\lambda\plus) \ge \tau\inv\inner{\gamma - \lambda\minus}{\lambda\minus - \lambda^+} + \frac 1{2\tau}||\lambda-\lambda^+||^2
\end{align*}
\end{lemma}

\begin{proof}
First, from \cref{opt:drs:1}, we have that 

\begin{align}
||\lambda^+ - \lambda\half|| \le \tau \sigma_G ||\lambda\minus - \lambda^+||.
\end{align}

\noindent We proceed by using the convexity of $H$ and \cref{prop:lipschitzcontgrad}.

\begin{align*}
H(\gamma) - H(\lambda\plus) &\ge H(\lambda\half) + \inner{\grad H(\lambda\half)}{\gamma - \lambda\half} \\
&\quad - \left(H(\lambda\half) + \inner{\grad H(\lambda\half)}{\lambda\plus - \lambda\half} + \frac{\sigma_H}{2}||\lambda\plus - \lambda\half||^2\right) \\
&= \inner{\grad H(\lambda\half)}{\gamma - \lambda\plus} - \frac{\sigma_H}{2}||\lambda\plus - \lambda\half||^2 \\
&\ge \inner{\grad H(\lambda\half)}{\gamma - \lambda\plus} - \frac{\sigma_H\tau^2\sigma_G^2}{2}||\lambda\minus - \lambda\plus||^2 \\
&\ge \inner{\grad H(\lambda\half)}{\gamma - \lambda\plus} - \frac{1}{2\tau}||\lambda\minus - \lambda\plus||^2 \\
\end{align*}

By the convexity of $G$, we have
\begin{align*}
G(\gamma) - G(\lambda^+) &\ge G(\lambda^+) + \inner{\gamma - \lambda^+}{\grad G(\lambda^+)} - G(\lambda^+) \\
&= \inner{\gamma - \lambda^+}{\grad G(\lambda^+)}. \\
\end{align*}

By adding the estimates together, and using \cref{opt:drs:2:alt}, we obtain

\begin{align*}
F(\gamma) - F(\lambda^+) &\ge \inner{\gamma - \lambda^+}{\grad H(\lambda\half) + \grad G(\lambda^+)} - \frac 1{2\tau} ||\lambda\minus - \lambda^+||^2 \\
&= \frac 1\tau \inner{\gamma - \lambda^+}{\lambda\minus - \lambda^+} - \frac 1{2\tau} ||\lambda\minus - \lambda^+||^2 \\
&= \frac 1\tau \inner{\gamma - \lambda\minus + \lambda\minus- \lambda^+}{\lambda\minus - \lambda^+} - \frac 1{2\tau} ||\lambda\minus - \lambda^+||^2 \\
&= \frac 1\tau \inner{\gamma - \lambda}{\lambda\minus - \lambda^+} + \frac 1{2\tau} ||\lambda\minus - \lambda^+||^2. \\
\end{align*}
\end{proof}


\section{Global Convergence Bounds for Unaccelerated DRS}

In this section we show that for DRS, we achieve the convergence
$F(\lambda_k) - F(\lambda^\star) \le O(1/k)$.

\begin{thrm}
Consider FDRS described by \cref{alg:drs}.  Suppose $H$ and $G$ satisfy assumption 1, and that $\tau^3 \le \frac 1{\sigma_H\sigma_G^2}$.  Then for $k > 1$ the sequence $\{\lambda_k\}$ satisfies

\begin{align*}
F(\lambda_k) - F(\lambda^\star) \le \frac {\tau ||\lambda^\star - \lambda_1||^2}{2(k-1)},
\end{align*}

\noindent where $\lambda^\star$ minimizes $F$.
\end{thrm}

\begin{proof}
We begin exploiting \cref{lem:1} with $\gamma = \lambda^\star$ and $\lambda\minus = \lambda_k$.  Then, $\lambda^+ = \lambda_{k+1}$.

\begin{align*}
2\tau(F(\lambda^\star) - F(\lambda_{k+1})) &\ge 2\inner{\lambda^\star - \lambda_k}{\lambda_k - \lambda_{k+1}} +  ||\lambda_k - \lambda_{k+1}||^2\\
&= ||\lambda^\star - \lambda_{k+1}||^2 - ||\lambda^\star - \lambda_k||^2.
\end{align*}

Summing over $k = 1, 2, \ldots, n-1$ yields

\begin{align}
2\tau\left((n-1)F(\lambda^\star) - \sum_{k=1}^{n-1} F(\lambda_{k+1}) \right) \ge ||\lambda^\star - \lambda_n||^2 - ||\lambda^\star - \lambda_1||^2. \label{eq:thrm1:1}
\end{align}

Using \cref{lem:1} again with $\lambda\minus = \gamma = \lambda_k$ gives

\begin{align}
2\tau \left(F(\lambda_k) - F(\lambda_{k+1})\right) \ge ||\lambda_k - \lambda_{k+1}||^2.\label{eq:thrm1:2}
\end{align}

Multiplying \cref{eq:thrm1:2} by $k-1$, and summing over $k = 1, 2, \ldots, n-1$ produces

\begin{align}
2\tau \left(\sum_{k=1}^{n-1} (k-1)\left[F(\lambda_k) - F(\lambda_{k+1})\right]\right) \ge \sum_{k=1}^{n-1} (k-1)||\lambda_k - \lambda_{k+1}||^2.\label{eq:thrm1:3}
\end{align}

The telescoping sum on the left of \cref{eq:thrm1:3} gives

\begin{align}
2\tau \left(\sum_{k=1}^{n-1} F(\lambda_{k+1}) - (n-1)F(\lambda_n)\right) \ge \sum_{k=1}^{n-1} (k-1)||\lambda_k - \lambda_{k+1}||^2.\label{eq:thrm1:4}
\end{align}

Adding \cref{eq:thrm1:1} and \cref{eq:thrm1:4} yields the bound

\begin{align}
2\tau(n-1)\left[F(\lambda^\star) - F(\lambda_n)\right] &\ge ||\lambda^\star - \lambda_n||^2 - ||\lambda^\star - \lambda_1||^2 +  \sum_{k=1}^{n-1} (k-1)||\lambda_k - \lambda_{k+1}||^2 \\
&\ge - ||\lambda^\star - \lambda_1||^2.
\end{align}

It follows that

\begin{align*}
F(\lambda_n) - F(\lambda^\star) \le \frac{||\lambda^\star - \lambda_1||^2}{2\tau(n-1)}.
\end{align*}
\end{proof}

%\begin{lemma}
%Assume that Assumption 1 holds.  We have the following rates of convergence on the residual:
%\begin{align*}
%||r_k||^2 := \le O(1/k).
%\end{align*}
%\end{lemma}

\section{Fast Douglas Rachford Splitting}

In this section, we consider a Nesterov like acceleration of the DRS algorithm, which we term Fast Douglas-Rachford Splitting (FDRS).

\begin{algorithm}
\caption{FDRS}
\label{alg:fdrs}
\begin{algorithmic}[1]
\REQUIRE $\alpha_0 = 1, \lambda_{-1} = \hat\lambda_0 \in R^N$
\FOR{$k = 0,1,2,\ldots$}
\STATE $\lambda_k = \texttt{DRSstep}(\hat\lambda_k)$
\STATE $\alpha_{k+1} = \frac {1+\sqrt{1+4\alpha_k^2}}2$ \\
\STATE $\hat\lambda_{k+1} = \lambda_k + \frac {\alpha_k - 1}{\alpha_{k+1}}(\lambda_k - \lambda_{k-1})$
\ENDFOR
\end{algorithmic}
\end{algorithm}

The optimality conditions for the above algorithm are:

\begin{align}
\hat\lambda_k - \lambda_{k+1/2} &= \tau \left(\grad G(\hat\lambda_k) + \grad H(\lambda_{k+1/2}) \right) \label{opt:fdrs:1}\\
\lambda_k - \lambda_{k+1/2} &= \tau \left(\grad G(\hat\lambda_k) - \grad G(\lambda_k)\right) \label{opt:fdrs:2} \\
\hat\lambda_k - \lambda_k &= \tau \left(\grad G(\lambda_k) + \grad H(\lambda_{k+1/2})\right) \text{(alternate)}. \label{opt:fdrs:2:alt}
\end{align}

Define
\begin{align*}
s_k = \alpha_k\lambda_k - (\alpha_k - 1)\lambda_{k-1} - \lambda^\star.
\end{align*}

\begin{lemma}
\label{lem:2}
Let $\lambda_k$, $\hat\lambda_k$ and $\alpha_k$ be defined as in \cref{alg:fdrs}.  Then

\begin{align*}
s_{k+1} = s_k + \alpha_{k+1}(\lambda_{k+1} - \hat\lambda_{k+1}).
\end{align*}
\end{lemma}

\begin{proof}
First note that from the definition of $\hat\lambda_{k+1}$ we have the identity

\begin{align}
\alpha_{k+1} (\hat\lambda_{k+1} - \lambda_k) = (\alpha_k - 1)(\lambda_k - \lambda_{k-1}). \label{eq:lem:2:1}
\end{align}

Using this observation in the definition of $s_k$ produces

\begin{align*}
s_{k+1} &= \alpha_{k+1}\lambda_{k+1} - (\alpha_{k+1} - 1)\lambda_k - \lambda^\star \\
&= \lambda_k - \lambda^\star + \alpha_{k+1}(\lambda_{k+1} - \lambda_k) \\
&= \lambda_k - (\alpha_k - 1)\lambda_{k-1} - \lambda^\star + \alpha_{k+1}(\lambda_{k+1} - \lambda_k) + (\alpha_k - 1)\lambda_{k-1}\\
&= \alpha_k\lambda_k - (\alpha_k - 1)\lambda_{k-1} - \lambda^\star + \alpha_{k+1}(\lambda_{k+1} - \lambda_k) - (\alpha_k - 1)(\lambda_k - \lambda_{k-1})\\
&= s_k + \alpha_{k+1}(\lambda_{k+1} - \lambda_k) - (\alpha_k - 1)(\lambda_k - \lambda_{k-1})\\
&= s_k + \alpha_{k+1}(\lambda_{k+1} - \lambda_k) - \alpha_{k+1}(\hat\lambda_{k+1} - \lambda_k)\\
&= s_k + \alpha_{k+1}(\lambda_{k+1} - \hat\lambda_{k+1}).\\
\end{align*}
\end{proof}

\noindent \Cref{lem:1,lem:2} combine for the following estimate.

\begin{lemma}
\label{lem:3}
Suppose that $H$ and $G$ satisfy Assumption 1 and that $G$ satisfies Assumption 2.  The iterates generated by \cref{alg:fdrs} without restart and the sequence $\{s_k\}$ obey the following relation:

\begin{align*}
||s_{k+1}||^2 - ||s_k||^2 \le 2\alpha_k^2\tau \left(F(\lambda^\star) - F(\lambda_k)\right) - 2\alpha_{k+1}^2 \tau\left(F(\lambda_{k+1}) - F(\lambda^\star)\right).
\end{align*}
\end{lemma}

\begin{proof}
To begin we apply \cref{lem:1} with $\gamma = \lambda_k$ and $\lambda\minus = \hat\lambda_{k+1}$, which gives $\lambda^+ = \lambda_{k+1}$.  We then have the following bound:

\begin{align}
\label{eq:lem:3:1}
F(\lambda_k) - F(\lambda_{k+1}) \ge \frac 1\tau\inner{\hat\lambda_{k+1} - \lambda_k}{\lambda_{k+1} - \hat\lambda_{k+1}} + \frac 1{2\tau}||\lambda_{k+1} - \hat\lambda_{k+1}||^2.
\end{align}

\noindent Applying \cref{lem:1} again with $\gamma = \lambda^\star$ and $\lambda\minus = \hat\lambda_{k+1}$ yields

\begin{align}
\label{eq:lem:3:2}
F(\lambda^\star) - F(\lambda_{k+1}) \ge \frac 1\tau\inner{\hat\lambda_{k+1} - \lambda^\star}{\lambda_{k+1} - \hat\lambda_{k+1}} + \frac 1{2\tau}||\lambda_{k+1} - \hat\lambda_{k+1}||^2.
\end{align}

\noindent Next, from \cref{lem:2}, we have

\begin{align*}
||s_{k+1}||^2 &= ||s_k + \alpha_{k+1}(\lambda_{k+1} - \hat\lambda_{k+1})||^2 \\
&= ||s_k||^2 + 2\alpha_{k+1}\inner{s_k}{\lambda_{k+1} - \hat\lambda_{k+1}} + \alpha_{k+1}^2||\lambda_{k+1} - \hat\lambda_{k+1}||^2. \\
\end{align*}

\noindent Rearranging this yields

\begin{align*}
||s_{k+1}||^2 - ||s_k||^2 &=  2\alpha_{k+1}\inner{s_k}{\lambda_{k+1} - \hat\lambda_{k+1}} + \alpha_{k+1}^2||\lambda_{k+1} - \hat\lambda_{k+1}||^2 \\
&= 2\alpha_{k+1}\inner{\alpha_k\lambda_k - (\alpha_k - 1)\lambda_{k-1} - \lambda^\star}{\lambda_{k+1} - \hat\lambda_{k+1}} \\
&\quad + \alpha_{k+1}^2||\lambda_{k+1} - \hat\lambda_{k+1}||^2 \\
&= 2\alpha_{k+1}\inner{\alpha_{k+1}\hat\lambda_{k+1} + (1 - \alpha_{k+1})\lambda_k - \lambda^\star}{\lambda_{k+1} - \hat\lambda_{k+1}} \\
&\quad + \alpha_{k+1}^2||\lambda_{k+1} - \hat\lambda_{k+1}||^2 \quad\text{from identity \cref{eq:lem:2:1}}\\ 
&= 2\alpha_{k+1}\inner{(\alpha_{k+1} - 1)(\hat\lambda_{k+1} - \lambda_k) + \hat\lambda_{k+1} - \lambda^\star}{\lambda_{k+1} - \hat\lambda_{k+1}} \\
&\quad + \alpha_{k+1}^2||\lambda_{k+1} - \hat\lambda_{k+1}||^2 \\
&= 2\alpha_{k+1}(\alpha_{k+1} - 1)\inner{\hat\lambda_{k+1} - \lambda_k}{\lambda_{k+1} - \hat\lambda_{k+1}} \\
&\quad + 2\alpha_{k+1}\inner{\hat\lambda_{k+1} - \lambda^\star}{\lambda_{k+1} - \hat\lambda_{k+1}} + \alpha_{k+1}^2||\lambda_{k+1} - \hat\lambda_{k+1}||^2 \\
&= 2\alpha_{k+1}(\alpha_{k+1} - 1)\left(\inner{\hat\lambda_{k+1} - \lambda_k}{\lambda_{k+1} - \hat\lambda_{k+1}} + \frac 12 ||\lambda_{k+1} - \hat\lambda_{k+1}||^2\right) \\
&\quad + 2\alpha_{k+1}\left(\inner{\hat\lambda_{k+1} - \lambda^\star}{\lambda_{k+1} - \hat\lambda_{k+1}} + \frac 12 ||\lambda_{k+1} - \hat\lambda_{k+1}||^2\right). \\
\end{align*}

\noindent Observe that $\alpha_{k+1} = \alpha_{k+1}^2 - \alpha_k^2$.  Using this, and the estimates \cref{eq:lem:3:1} and \cref{eq:lem:3:2} with the above equality yields

\begin{align*}
||s_{k+1}||^2 - ||s_k||^2 &\le 2\alpha_{k+1}\tau(\alpha_{k+1} - 1)(F(\lambda_k) - F(\lambda_{k+1})) \\
&\quad + 2\alpha_{k+1}\tau(F(\lambda^\star) - F(\lambda_{k+1})) \\
&= 2\alpha_{k+1}(\alpha_{k+1} - 1)\tau F(\lambda_k) + 2\alpha_{k+1}\tau F(\lambda^\star) \\
&\quad - 2\alpha_{k+1}^2\tau F(\lambda_{k+1})\\
&= 2\alpha_k^2\tau F(\lambda_k) + 2\left(\alpha_{k+1}^2-\alpha_k^2\right)\tau F(\lambda^\star) \\
&\quad - 2\alpha_{k+1}^2\tau F(\lambda_{k+1})\\
&= 2\alpha_k^2\tau\left[F(\lambda_k) - F(\lambda^\star)\right] - 2\alpha_{k+1}^2\tau\left[F(\lambda_{k+1}) - F(\lambda^\star)\right].\\
\end{align*}

\end{proof}

\begin{thrm}
Suppose that $H$ and $G$ satisfy Assumption 1 and that $G$ satisfies Assumption 2.  The iterates $\lambda_k$ generated by \cref{alg:fdrs} without restart satisfy
\begin{align*}
F(\lambda_k) - F(\lambda^\star) \le \frac{2\tau||\hat\lambda_0 - \lambda^\star||^2}{(k+2)^2}.
\end{align*}
\end{thrm}

\begin{proof}
\Cref{lem:3} can be rearranged as

\begin{align*}
||s_{k+1}||^2 + 2\alpha_{k+1}^2 \tau\left[F(\lambda_{k+1}) - F(\lambda^\star)\right] \le ||s_k||^2 + 2\alpha_k^2\tau \left[F(\lambda_k) - F(\lambda^\star)\right].
\end{align*}

Recalling that $\alpha_0 = 1$, induction then implies that 

\begin{align}
\label{eq:thrm:2:1}
||s_{k+1}||^2 + 2\alpha_{k+1}^2 \tau\left[F(\lambda_{k+1}) - F(\lambda^\star)\right] \le ||s_0||^2 + 2\tau \left[F(\lambda_0) - F(\lambda^\star)\right],
\end{align}

Then applying \cref{eq:lem:3:2} with $k = 0$ gives

\begin{align}
\label{eq:thrm:2:2}
F(\lambda^\star) - F(\lambda_0) \ge \frac 1\tau\inner{\hat\lambda_0 - \lambda^\star}{\lambda_0 - \hat\lambda_0} + \frac 1{2\tau}||\lambda_0 - \hat\lambda_0||^2 = \frac 1{2\tau}\left(||\lambda_0 - \lambda^\star||^2 - ||\hat\lambda_0 - \lambda^\star||^2\right).
\end{align}

Applying estimates \cref{eq:thrm:2:1} and \cref{eq:thrm:2:2} to \cref{lem:3} we have 

\begin{align*}
2\alpha_{k+1}^2\tau(F(\lambda_{k+1}) - F(\lambda^\star)) &\le ||s_k||^2 - ||s_{k+1}||^2 + 2\alpha_k^2\tau\left[F(\lambda_k) - F(\lambda^\star)\right] \\
&\le ||s_k||^2 + 2\alpha_k^2\tau\left[F(\lambda_k) - F(\lambda^\star)\right] \\
&\le ||s_0||^2 + 2\tau\left[F(\lambda_0) - F(\lambda^\star)\right] \\
&= ||\lambda_0 - \lambda^\star||^2 + 2\tau\left[F(\lambda_0) - F(\lambda^\star)\right] \\
&\le ||\lambda_0 - \lambda^\star||^2 + ||\hat\lambda_0 - \lambda^\star||^2 - ||\lambda_0 - \lambda^\star||^2 \\
&= ||\hat\lambda_0 - \lambda^\star||^2.
\end{align*}

It then follows that

\begin{align*}
F(\lambda_k) - F(\lambda^\star) &\le \frac{||\hat\lambda_0 - \lambda^\star||^2}{2\alpha_k^2\tau}.
\end{align*}

Using the observation that $\alpha_k > \alpha_{k-1} + \frac 12 > 1 + \frac k2$ gives

\begin{align*}
F(\lambda_k) - F(\lambda^\star) &\le \frac{2||\hat\lambda_0 - \lambda^\star||^2}{(k+2)^2\tau}.
\end{align*}

\end{proof}

\section{Fast Douglas Rachford Splitting for Non-Differential Problems}

We consider a variant of \cref{alg:fdrs} in which we impose a restart
condition, which reverts the algorithm to the traditional Douglas
Rachford Splitting method in certain cases, which ensures convergence.
The restart rule relies on a combined residual, which bounds the
residual for \cref{prb:1}:

\begin{align}
\begin{split}
r_k &= ||\grad G(\lambda_k) + \grad H(\lambda_k)|| \\
&\le || \grad G(\lambda_k) + \grad H(\lambda_{k+1/2}) || \\
&\quad + || \grad H(\lambda_k) - \grad H(\lambda_{k+1/2}) || =: c_k.
\end{split}
\end{align}

\noindent The algorithm is then given by \cref{alg:fdrsRestart}.
Notice that Lines 1-2 produce $\lambda_{k+1} =
\texttt{DRSstep}(\hat\lambda_k)$, as in \cref{alg:fdrs}.


\begin{algorithm}
\caption{FDRS with Restart}
\label{alg:fdrsRestart}
\begin{algorithmic}[1]
\REQUIRE $\alpha_0 = 1, \lambda_{-1} = \hat\lambda_0 \in R^N$
\FOR{$k = 0,1,2,\ldots$}
\STATE $\lambda_{k+1/2} = J_{\tau \grad H}(\hat\lambda_k - \tau \grad G(\hat\lambda)) = \argmin{\lambda} \tau H(\lambda) + \frac 12||\lambda - \hat\lambda_k + \tau \grad G(\hat\lambda_k)||^2$
\STATE $\lambda_{k+1} = J_{\tau \grad G}(\lambda_{k+1/2} + \tau \grad G(\hat\lambda)) = \argmin{\lambda} \tau G(\lambda) + \frac 12||\lambda - \lambda_{k+1/2} - \tau \grad G(\hat\lambda_k)||^2$
  \STATE $c_k = ||\grad G(\lambda_{k+1}) + \grad H(\lambda_{k+1/2})|| + ||\grad H(\lambda_{k+1}) - \grad H(\lambda_{k+1/2})||$
  \IF{$c_k < \eta c_{k-1}$}
  \STATE $\alpha_{k+1} = \frac {1+\sqrt{1+4\alpha_k^2}}2$ \\
  \STATE $\hat\lambda_{k+1} = \lambda_k + \frac {\alpha_k - 1}{\alpha_{k+1}}(\lambda_k - \lambda_{k-1})$
  \ELSE
  \STATE $\alpha_{k+1} = 1$, $\lambda_{k+1/2} = \lambda_{k-1}$
  \STATE $c_k \leftarrow \eta\inv c_{k-1}$
  \ENDIF
\ENDFOR
\end{algorithmic}
\end{algorithm}

\begin{lemma}
\label{lem:4}
The iterates $\{\lambda_k, \lambda_{k+1/2}\}$ produced in \cref{alg:fdrsRestart} satisfy
\begin{align}
\begin{split}
|| \grad G(\lambda_{k+1}) + \grad H(\lambda_{k+3/2}) || + || \grad H(\lambda_{k+1}) - \grad H(\lambda_{k+3/2}) || \\
\le || \grad G(\lambda_k) + \grad H(\lambda_{k+1/2}) || + || \grad H(\lambda_k) - \grad H(\lambda_{k+1/2}) ||.
\end{split}
\end{align}
\end{lemma}

\begin{proof}
Blah blah blah cite Yuan and He.  (But they do it for ADMM....)
\end{proof}

\begin{thrm}
For convex $H$ and $G$, \cref{alg:fdrsRestart} converges:

\begin{align}
\lim_{k\rightarrow \infty} c_k = 0.
\end{align}
\end{thrm}
\begin{proof}
We begin with some terminology. Each iteration of
\cref{alg:fdrsRestart} is of one of three types: 

\begin{enumerate}
\item A “restart” iteration occurs when the inequality in Step 5 of
  the algorithm is not satisfied.
\item A “non-accelerated” iteration occurs immediately after
a “restart” iteration. On such iterations $\alpha_k = 1$, and so the
acceleration (Lines 6-7) of \cref{alg:fdrsRestart} are inactivated making the
iteration equivalent to the original ADMM. 
\item An “accelerated” iteration is any iteration that is not
  “restart” or “unaccelerated.”  On such iterations, Lines 6-7 of the
  algorithm are invoked and $\alpha_k > 1$.  
\end{enumerate}
Suppose that a restart occurs at iteration k. Then the value
$\lambda_k$ are returned to their values at iteration $k-1$, which has
combined residual $c_{k-1}$. We also set $\alpha_{k+1} = 1$, making
iteration $k + 1$ an unaccelerated iteration.  By \cref{lem:4} the
combined residual is non-increasing on this iteration and so $c_{k+1}
\le c_{k-1}$.  Note that on accelerated steps, the combined residual
decreases by at least a factor of $\eta$. It follows that the combined
residual satisfies 
\begin{align}
c_k \le c_0\eta^{\hat k},
\end{align} 
\noindent where $\hat k$ denotes the number of accelerated steps that
have occurred within the first $k$ iterations.  

Clearly, if the number of accelerated iterations is infinite, then we
have $c_k \rightarrow 0$ as $k \rightarrow \infty$. In the case that
the number of accelerated iterations is finite, then after the final
accelerated iteration, each pair of restart and unaccelerated
iterations is equivalent to a single iteration of the original
unaccelerated DRS (\cref{alg:drs}) for which convergence is known
cite[]. 􏰏
\end{proof}

While \cref{alg:fdrsRestart} enables us to extend accelerated DRS to
non-differential problems, our theoretical results are weaker in this
case because we cannot guarantee a convergence rate as we did under
\cref{assum:1}. Nevertheless, the empirical behavior of the restart
method (Algorithm 8) is superior to that of the original ADMM
(Algorithm 1), even in the case of strongly convex functions. Similar
results have been observed for restarted variants of other accelerated
schemes [35].

\section{Backtracking}

State the line search method.  I (TOM) will do some experiments to
figure out the best way to do this, but try to structure the Lemma in
the preliminaries section so the Lipschitz gradient part is separate
from the main inequality Lemma.

\section{Relation to Other Work}
 \begin{itemize}
 \item Global Convergence rates for DRS have also been proved by Binsheng He (I'll dig this paper up for you), but without any acceleration.
 \item This is similar to the ADMM method, however our results are a stronger because of the line search - we can gaurantee convergence for non-differentiable problems (i.e. the duals of weakly convex problems that would normally be solved with ADMM)
 \end{itemize}

\section{Numerical Results}
 \begin{itemize}
 \item Do some simple quadratic programming examples
 \item We need to come up with a few more examples of things that DRS is good for.  There's tons of these, but we have to pick a few that are easy to code.
 \end{itemize}
\end{document}
